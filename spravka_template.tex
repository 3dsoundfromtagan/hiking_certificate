\documentclass[a4paper, 12pt]{extreport}
\usepackage[T2A]{fontenc} % выбор внутренней кодировки 
\usepackage[english, russian]{babel} % выбор языка
\usepackage[left=10mm, right=10mm, top=10 mm, bottom=20mm]{geometry} % поля
\setlength{\parindent}{0cm} % длина отступа первой строки абзаца
\linespread{1} % междустрочный интервал
\usepackage{makecell}
\usepackage{mathtools}
\newcommand{\su}[2]{$\underbracket[0.3pt][0pt]{\textrm{#1}}_\textrm{{#2}}$}

\begin{document}
\pagestyle{empty}
\begin{tabular}{>{\centering\arraybackslash}m{0.13\linewidth}>{\centering\arraybackslash}p{0.87\linewidth}}
		\includegraphics[width=1\linewidth]{FSTR_logo} 
	& \begin{tabular}{>{\centering\arraybackslash}m{0.9\linewidth}}
	   {\large \textbf{Федерация спортивного туризма Московской области}}  \\
		\rule{0pt}{30pt}
		\Large\textbf{СПРАВКА}	\\
		 {\large \textbf{о зачёте прохождения туристкого спортивного маршрута}} \\
	\end{tabular} \\
\end{tabular}
\vspace{-1cm}
\begin{flushright}
	\su{\large 01/2099}{№ маршрутной книжки}
\end{flushright}
\vspace{-0.5cm}

\begin{center}
	 {\large Выдана туристу \su{\Large\textbf{\textit{Жумарову Айсбайлю Бергшрундовичу}}}{фамилия, имя, отчество}}
\end{center}

\noindent
\resizebox{\textwidth}{!}{%
	\begin{tabular}{|>{\centering\arraybackslash}m{0.1\linewidth}|>{\centering\arraybackslash}m{0.13\linewidth}|>{\centering\arraybackslash}m{0.1\linewidth}|>{\centering\arraybackslash}m{0.12\linewidth}|>{\centering\arraybackslash}m{0.1\linewidth}|>{\centering\arraybackslash}m{0.11\linewidth}|>{\centering\arraybackslash}m{0.11\linewidth}|>{\centering\arraybackslash}m{0.13\linewidth}|}
		\hline 
		\small \makecell{Год,\\месяц} &
		\small \text{Район} &
		\small \makecell{Вид\\туризма} &
		\small \makecell{Способ\\передвиже-\\ния} &
		\small \makecell{Протя-\\жённость,\\км} &
		\small \makecell{Продолжи-\\тельность,\\дней} &
		\small \makecell{Категория\\сложности} &
		\small \makecell{Руководство\\или участие} \\
		\hline			
		
		\textbf{\textit{2099 август}}	&	\textbf{\textit{Западный Кавказ, Гвандра}}	&	\textbf{\textit{горный}}	&	\textbf{\textit{пешком}}	&	\textbf{\textit{111.0}}	& \textbf{\textit{13}} & \textbf{\textit{первая}} & \textbf{\textit{участие}} \\
		\hline	
	\end{tabular}%
}
	
\vspace{0.5cm}
\begin{small}
	Подробная нитка маршрута с указанием начального, конечного пункта и определяющих категорию сложности препятствий:
\end{small}
\vspace{0.5cm}

\normalsize \textit{д.р. Учкулан~--- д.р. Кичкинакол Уллукёльский~--- \textbf{пер. Уллукёль Восточный (1А, 3050)}~--- т/б <<Глобус>>~--- д.р.Гондарай~--- д.р. Джалпаккол~--- \textbf{пер. Джалпаккол Сев. (1А, 3411)}~--- д.р. Мырды~--- а/л<<Узункол>>~--- д.р.Кичкинекол~--- д.р. Таллычат~--- \textbf{пер. Кичкинекол Малый (1А, 3206)}~--- д.р. Чунгур-Джар~--- \textbf{пер. Перемётный (1А, 3255)}~--- д.р. Танышхан~--- д.р. Чиринкол~--- стоянка <<Гвандра>>~--- д. р. Кубань~--- \textbf{пер. Хотютау (1А, 3546)}~--- лед. Большой Азау~--- ст.~<<Кругозор>>} 

\newpage

\begin{center}
	{ \large \textbf{Препятствия, определяющие категорию сложности маршрута}}
\end{center}

\resizebox{\textwidth}{!}{%
	\begin{tabular}{|>{\centering\arraybackslash}m{0.12\linewidth}|>{\centering\arraybackslash}m{0.2\linewidth}|>{\centering\arraybackslash}m{0.12\linewidth}|>{\raggedright\arraybackslash}m{0.45\linewidth}|}
		\hline 
		\small \makecell{Вид\\препятствия} &
		\small \text{Название} &
		\small \makecell{Категория\\сложности} &
		\small \makecell{Характеристика препятствия} \\
		\hline			
		
		\textit{перевал}	&	\textit{Уллу-Кёль Восточный}	&	\textit{1А}	&	\textit{высота 3050~м, скально-снежно-осыпной}	 \\
		\hline

		\textit{перевал}	&	\textit{Джалпаккол Северный}	&	\textit{1А}	&	\textit{высота 3411~м, ледово-скально-осыпной}	 \\
		\hline	
		
		\textit{перевал}	&	\textit{Кичинекол малый}	&	\textit{1А}	&	\textit{высота 3206~м, осыпной}	 \\
		\hline	
		
		\textit{перевал}	&	\textit{Перемётный}	&	\textit{1А}	&	\textit{высота 3255~м, осыпной}	 \\
		\hline	
		
		\textit{перевал}	&	\textit{Хотютау}	&	\textit{1А}	&	\textit{высота 3546~м, скально-осыпной}	 \\
		\hline	
	\end{tabular}%
}
\vspace{0.5cm}

\begin{small}
	Особые отметки МКК (изменение категории сложности и т.п.): ~\hrulefill
	
	\underline{\hspace{\textwidth}}            
\end{small}

\small Замечание руководителя о маршруте и участнике группы: ~\hrulefill


\underline{\hspace{\textwidth}}   

\underline{\hspace{\textwidth}}     

\vspace{0.5cm}

\normalsize

	\begin{tabular}{>{\raggedright\arraybackslash}m{0.32\linewidth}>{\raggedright\arraybackslash}m{0.34\linewidth}>{\raggedright\arraybackslash}m{0.28\linewidth}}
		Подпись /ФИО руководителя/: & \su{\hspace{2cm}}{подпись} \quad /Руководов Р.Р./  &  \\
		
		\textbf{Председатель МКК:}	&	\su{\hspace{2cm}}{подпись}	&	\makecell[l]{\small 			
																														Отметка о сдаче и местона-\\хождении
																														отчёта о походе,\\
																														инв. №~\hrulefill\\
																														в библиотеке:~\hrulefill \\
																														дата:~\hrulefill
																													}\\
			Штамп МКК: & \underline{<<\qquad>>}~\underline{\hspace{3cm}}~\underline{\hspace{1cm}~г.}  &  \\
	\end{tabular}%



\vspace{0.5cm}

\end{document}